%
% This is the LaTeX template file for lecture notes for CS294-8,
% Computational Biology for Computer Scientists.  When preparing 
% LaTeX notes for this class, please use this template.
%
% To familiarize yourself with this template, the body contains
% some examples of its use.  Look them over.  Then you can
% run LaTeX on this file.  After you have LaTeXed this file then
% you can look over the result either by printing it out with
% dvips or using xdvi.
%
% This template is based on the template for Prof. Sinclair's CS 270.

\documentclass{article}
\usepackage{graphics}
\usepackage{multicol}
\usepackage{tikz}
\usepackage{clrscode3e}
\usepackage{enumitem}
\usepackage{amsmath}
\usepackage{amsthm}
\usepackage{amssymb}

\setlength{\oddsidemargin}{0 in}
\setlength{\evensidemargin}{0 in}
\setlength{\topmargin}{-0.6 in}
\setlength{\textwidth}{6.5 in}
\setlength{\textheight}{8.5 in}
\setlength{\headsep}{0.75 in}
\setlength{\parindent}{0 in}
\setlength{\parskip}{0.1 in}

%
% The following commands set up the lecnum (lecture number)
% counter and make various numbering schemes work relative
% to the lecture number.
%
\newcounter{lecnum}
\renewcommand{\thepage}{\thelecnum-\arabic{page}}
\renewcommand{\thesection}{\thelecnum.\arabic{section}}
\renewcommand{\theequation}{\thelecnum.\arabic{equation}}
\renewcommand{\thefigure}{\thelecnum.\arabic{figure}}
\renewcommand{\thetable}{\thelecnum.\arabic{table}}

%
% The following macro is used to generate the header.
%
\newcommand{\lecture}[4]{
   \pagestyle{myheadings}
   \thispagestyle{plain}
   \newpage
   \setcounter{lecnum}{#1}
   \setcounter{page}{1}
   \noindent
   \begin{center}
   \framebox{
      \vbox{\vspace{2mm}
    \hbox to 6.28in { {\bf CSC2420 - Algorithm Design, Analysis and Theory
                        \hfill Fall 2022} }
       \vspace{4mm}
       \hbox to 6.28in { {\Large \hfill Lecture #1: #2  \hfill} }
       \vspace{2mm}
       \hbox to 6.28in { {\it Lecturer: #3 \hfill Scribe: #4} }
      \vspace{2mm}}
   }
   \end{center}
   \markboth{Lecture #1: #2}{Lecture #1: #2}
   % {\bf Disclaimer}: {\it These notes have not been subjected to the
   % usual scrutiny reserved for formal publications.  They may be distributed
   % outside this class only with the permission of the Instructor.}
   \vspace*{4mm}
}

%
% Convention for citations is authors' initials followed by the year.
% For example, to cite a paper by Leighton and Maggs you would type
% \cite{LM89}, and to cite a paper by Strassen you would type \cite{S69}.
% (To avoid bibliography problems, for now we redefine the \cite command.)
% Also commands that create a suitable format for the reference list.
\renewcommand{\cite}[1]{[#1]}
\def\beginrefs{\begin{list}%
        {[\arabic{equation}]}{\usecounter{equation}
         \setlength{\leftmargin}{2.0truecm}\setlength{\labelsep}{0.4truecm}%
         \setlength{\labelwidth}{1.6truecm}}}
\def\endrefs{\end{list}}
\def\bibentry#1{\item[\hbox{[#1]}]}

%Use this command for a figure; it puts a figure in wherever you want it.
%usage: \fig{NUMBER}{SPACE-IN-INCHES}{CAPTION}
\newcommand{\fig}[3]{
			\vspace{#2}
			\begin{center}
			Figure \thelecnum.#1:~#3
			\end{center}
	}
% Use these for theorems, lemmas, proofs, etc.
\newtheorem{theorem}{Theorem}[lecnum]
\newtheorem{lemma}[theorem]{Lemma}
\newtheorem{proposition}[theorem]{Proposition}
\newtheorem{claim}[theorem]{Claim}
\newtheorem{corollary}[theorem]{Corollary}
\newtheorem{definition}[theorem]{Definition}
\newtheorem{problem}[theorem]{Problem}
\newtheorem{conjecture}[theorem]{Conjecture}
\renewenvironment{proof}{{\bf Proof:}}{\hfill\rule{2mm}{2mm}}

\newcommand{\Exp}{\mathbb{E}}
\newcommand{\Var}{\mathrm{Var}}


\begin{document}
%FILL IN THE RIGHT INFO.
%\lecture{**LECTURE-NUMBER**}{**DATE**}{**LECTURER**}{**SCRIBE**}
\lecture{4}{Priority Algorithms}{Allan Borodin}{Kevin Gao}

Recall the template for the fixed order priority algorithm template:

\begin{codebox}
    \li $\mathcal{J} =$ set of all possible inputs
    \li $\preceq\,\, = $ a total ordering on $\mathcal{J}$ (typically induced by a function $f$)
    \li $\mathcal{I} \subset \mathcal{J} = $ actual input to the algorithm
    \li $S = \emptyset$ \RComment{items already examined by the algorithm}
    \li $i = 0$
    \li \While $\mathcal{I} - S \neq \emptyset$ \Do
        \li $i = i + 1$
        \li $\mathcal{I} = \mathcal{I} - S$
        \li $I_i = \min_{\preceq} \{ I \in \mathcal{I} \}$ \RComment{select min element based on the ordering $\preceq$}
        \li make an irrevocable decision $D_i$ concerning $I_i$ 
        \li $S = S \cup \{I_i\}$
    \End      
\end{codebox}

and the adaptive priority algorithm:

\begin{codebox}
    \li $\mathcal{J} =$ set of all possible inputs
    \li $\mathcal{I} \subset \mathcal{J} = $ actual input to the algorithm
    \li $S = \emptyset$ \RComment{items already examined by the algorithm}
    \li $i = 0$
    \li \While $\mathcal{I} - S \neq \emptyset$ \Do
        \li $i = i + 1$
        \li $\preceq_i\,\, = $ a total ordering on $\mathcal{J}$ (typically induced by a function $f_i$)
        \li $\mathcal{I} = \mathcal{I} - S$
        \li $I_i = \min_{\preceq_i} \{ I \in \mathcal{I} \}$ \RComment{select min element based on the ordering $\preceq_i$}
        \li make an irrevocable decision $D_i$ concerning $I_i$ 
        \li $S = S \cup \{I_i\}$
        \li $\mathcal{J} = \mathcal{J} - \{I \in \mathcal{I} \mid I \preceq_i I_i \}$
    \End      
\end{codebox}

\section{Inapproximations for Deterministic Priority Algorithms}

Once we have a precise model, we can then argue that certain approximation bounds are impossible within this model. We first consider the weighted interval selection problem.

For the interval selection problem, we have a set of intervals and each interval has a weight $w_j$. No priority algorithm can achieve a constant approximation. In an undergraduate course, we have shown that the unweighted interval selection problem can be optimally solved using a greedy algorithm by ordering the inputs by the earliest finishing time. We will prove a weaker result.

\begin{theorem}
    There is no priority approximation algorithm can achieve an approximation ratio better than 3.
\end{theorem}
\begin{proof}
    We use a charging argument. 
\end{proof}

\section{Set Cover}

In the set cover problem, we are given a collection $\mathcal{S} = \{S_1,\ldots,S_n\}$ of sets with $S_i \subseteq U$ for some universe $U$. In the weighted set cover problem, each set $S_i$ has a cost or weight $c(S_i)$. The objective is to find a minimum cost subcollection $\mathcal{S}'$ such that $\bigcup_{S \in \mathcal{S}} S = U$.

For the set cover problem, the ``natural adaptive greedy algorithm'' is essentially the best priority algorithm.

\begin{codebox}
    \Procname{$\proc{Greedy-Set-Cover}(\mathcal{S})$}
    \li $\mathcal{S}' = \emptyset$
    \li \While $U$ is uncovered \Do
        \li $j = \arg\min_{i} \{w(S_i) / |S_i \cap U|\}$
        \li $\mathcal{S}' = \mathcal{S}' \cup \{S_j\}$
        \li $U = U \setminus \{S_j\}$ 
\end{codebox}

\end{document}
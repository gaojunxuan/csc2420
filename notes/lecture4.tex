\input{scribe.tex}

\begin{document}
%FILL IN THE RIGHT INFO.
%\lecture{**LECTURE-NUMBER**}{**DATE**}{**LECTURER**}{**SCRIBE**}
\lecture{4}{Priority Algorithms}{Allan Borodin}{Kevin Gao}

Recall the template for the fixed order priority algorithm template:

\begin{codebox}
    \li $\mathcal{J} =$ set of all possible inputs
    \li $\preceq\,\, = $ a total ordering on $\mathcal{J}$ (typically induced by a function $f$)
    \li $\mathcal{I} \subset \mathcal{J} = $ actual input to the algorithm
    \li $S = \emptyset$ \RComment{items already examined by the algorithm}
    \li $i = 0$
    \li \While $\mathcal{I} - S \neq \emptyset$ \Do
        \li $i = i + 1$
        \li $\mathcal{I} = \mathcal{I} - S$
        \li $I_i = \min_{\preceq} \{ I \in \mathcal{I} \}$ \RComment{select min element based on the ordering $\preceq$}
        \li make an irrevocable decision $D_i$ concerning $I_i$ 
        \li $S = S \cup \{I_i\}$
    \End      
\end{codebox}

and the adaptive priority algorithm:

\begin{codebox}
    \li $\mathcal{J} =$ set of all possible inputs
    \li $\mathcal{I} \subset \mathcal{J} = $ actual input to the algorithm
    \li $S = \emptyset$ \RComment{items already examined by the algorithm}
    \li $i = 0$
    \li \While $\mathcal{I} - S \neq \emptyset$ \Do
        \li $i = i + 1$
        \li $\preceq_i\,\, = $ a total ordering on $\mathcal{J}$ (typically induced by a function $f_i$)
        \li $\mathcal{I} = \mathcal{I} - S$
        \li $I_i = \min_{\preceq_i} \{ I \in \mathcal{I} \}$ \RComment{select min element based on the ordering $\preceq_i$}
        \li make an irrevocable decision $D_i$ concerning $I_i$ 
        \li $S = S \cup \{I_i\}$
        \li $\mathcal{J} = \mathcal{J} - \{I \in \mathcal{I} \mid I \preceq_i I_i \}$
    \End      
\end{codebox}

\section{Inapproximations for Deterministic Priority Algorithms}

Once we have a precise model, we can then argue that certain approximation bounds are impossible within this model. We first consider the weighted interval selection problem.

For the interval selection problem, we have a set of intervals and each interval has a weight $w_j$. No priority algorithm can achieve a constant approximation. In an undergraduate course, we have shown that the unweighted interval selection problem can be optimally solved using a greedy algorithm by ordering the inputs by the earliest finishing time. We will prove a weaker result.

\begin{theorem}
    There is no priority approximation algorithm can achieve an approximation ratio better than 3.
\end{theorem}
\begin{proof}
    We use a charging argument. 
\end{proof}

\section{Set Cover}

In the set cover problem, we are given a collection $\mathcal{S} = \{S_1,\ldots,S_n\}$ of sets with $S_i \subseteq U$ for some universe $U$. In the weighted set cover problem, each set $S_i$ has a cost or weight $c(S_i)$. The objective is to find a minimum cost subcollection $\mathcal{S}'$ such that $\bigcup_{S \in \mathcal{S}} S = U$.

For the set cover problem, the ``natural adaptive greedy algorithm'' is essentially the best priority algorithm.

\begin{codebox}
    \Procname{$\proc{Greedy-Set-Cover}(\mathcal{S})$}
    \li $\mathcal{S}' = \emptyset$
    \li \While $U$ is uncovered \Do
        \li $j = \arg\min_{i} \{w(S_i) / |S_i \cap U|\}$
        \li $\mathcal{S}' = \mathcal{S}' \cup \{S_j\}$
        \li $U = U \setminus \{S_j\}$ 
\end{codebox}

\end{document}
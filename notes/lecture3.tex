%
% This is the LaTeX template file for lecture notes for CS294-8,
% Computational Biology for Computer Scientists.  When preparing 
% LaTeX notes for this class, please use this template.
%
% To familiarize yourself with this template, the body contains
% some examples of its use.  Look them over.  Then you can
% run LaTeX on this file.  After you have LaTeXed this file then
% you can look over the result either by printing it out with
% dvips or using xdvi.
%
% This template is based on the template for Prof. Sinclair's CS 270.

\documentclass{article}
\usepackage{graphics}
\usepackage{multicol}
\usepackage{tikz}
\usepackage{clrscode3e}
\usepackage{enumitem}
\usepackage{amsmath}
\usepackage{amsthm}
\usepackage{amssymb}

\setlength{\oddsidemargin}{0 in}
\setlength{\evensidemargin}{0 in}
\setlength{\topmargin}{-0.6 in}
\setlength{\textwidth}{6.5 in}
\setlength{\textheight}{8.5 in}
\setlength{\headsep}{0.75 in}
\setlength{\parindent}{0 in}
\setlength{\parskip}{0.1 in}

%
% The following commands set up the lecnum (lecture number)
% counter and make various numbering schemes work relative
% to the lecture number.
%
\newcounter{lecnum}
\renewcommand{\thepage}{\thelecnum-\arabic{page}}
\renewcommand{\thesection}{\thelecnum.\arabic{section}}
\renewcommand{\theequation}{\thelecnum.\arabic{equation}}
\renewcommand{\thefigure}{\thelecnum.\arabic{figure}}
\renewcommand{\thetable}{\thelecnum.\arabic{table}}

%
% The following macro is used to generate the header.
%
\newcommand{\lecture}[4]{
   \pagestyle{myheadings}
   \thispagestyle{plain}
   \newpage
   \setcounter{lecnum}{#1}
   \setcounter{page}{1}
   \noindent
   \begin{center}
   \framebox{
      \vbox{\vspace{2mm}
    \hbox to 6.28in { {\bf CSC2420 - Algorithm Design, Analysis and Theory
                        \hfill Fall 2022} }
       \vspace{4mm}
       \hbox to 6.28in { {\Large \hfill Lecture #1: #2  \hfill} }
       \vspace{2mm}
       \hbox to 6.28in { {\it Lecturer: #3 \hfill Scribe: #4} }
      \vspace{2mm}}
   }
   \end{center}
   \markboth{Lecture #1: #2}{Lecture #1: #2}
   % {\bf Disclaimer}: {\it These notes have not been subjected to the
   % usual scrutiny reserved for formal publications.  They may be distributed
   % outside this class only with the permission of the Instructor.}
   \vspace*{4mm}
}

%
% Convention for citations is authors' initials followed by the year.
% For example, to cite a paper by Leighton and Maggs you would type
% \cite{LM89}, and to cite a paper by Strassen you would type \cite{S69}.
% (To avoid bibliography problems, for now we redefine the \cite command.)
% Also commands that create a suitable format for the reference list.
\renewcommand{\cite}[1]{[#1]}
\def\beginrefs{\begin{list}%
        {[\arabic{equation}]}{\usecounter{equation}
         \setlength{\leftmargin}{2.0truecm}\setlength{\labelsep}{0.4truecm}%
         \setlength{\labelwidth}{1.6truecm}}}
\def\endrefs{\end{list}}
\def\bibentry#1{\item[\hbox{[#1]}]}

%Use this command for a figure; it puts a figure in wherever you want it.
%usage: \fig{NUMBER}{SPACE-IN-INCHES}{CAPTION}
\newcommand{\fig}[3]{
			\vspace{#2}
			\begin{center}
			Figure \thelecnum.#1:~#3
			\end{center}
	}
% Use these for theorems, lemmas, proofs, etc.
\newtheorem{theorem}{Theorem}[lecnum]
\newtheorem{lemma}[theorem]{Lemma}
\newtheorem{proposition}[theorem]{Proposition}
\newtheorem{claim}[theorem]{Claim}
\newtheorem{corollary}[theorem]{Corollary}
\newtheorem{definition}[theorem]{Definition}
\newtheorem{problem}[theorem]{Problem}
\newtheorem{conjecture}[theorem]{Conjecture}
\renewenvironment{proof}{{\bf Proof:}}{\hfill\rule{2mm}{2mm}}

\newcommand{\Exp}{\mathbb{E}}
\newcommand{\Var}{\mathrm{Var}}


\begin{document}
%FILL IN THE RIGHT INFO.
%\lecture{**LECTURE-NUMBER**}{**DATE**}{**LECTURER**}{**SCRIBE**}
\lecture{3}{Online Bipartite Matching}{Calum McRury}{Kevin Gao}

Recall that the ranking algorithm independently assigns a ranking for each offline vertex and match the online vertices based on the ranking of the available offline vertex as the online vertices arrive.

An intuitive interpretation of the matching problem is as follows. One side $U$ are the sellers, and the other side $V$ are the buyers. Each seller draws a price $x_u$ uniformly and indepdently from $[0,1]$. We as the ``government'', set an adjusted price $g(x_u)$ that the buyers end up seeing as the final pricetag. Then, when the buyer $v \in V$ arrives, the buyer greedily choose the cheapest item that it can buy (in $N(v)$), contributing to the overall utility by $(1-g(x_u))$. In the analysis, we adjust each both the adjusted price and the utility gain by a factor of $1/F$ where $F = 1-1/e$. Setting $\alpha_u = g(x_u)/F$ and for $v \in N(u)$, $\beta_v = (1-g(_u))/F$, we wish to maximize
\label{eq:online-matching-objective}
\begin{equation}
    |\mathcal{M}| = F \cdot \left[ \sum_{u \in U} \alpha_u + \sum_{v \in V} \beta_v \right].
\end{equation}

\begin{theorem}
    Ranking returns a matching $\mathcal{M}$ of $G=(U,V,E)$ such that
    $$
    \mathbb{E}[|\mathcal{M}|] \geq F \cdot \mathrm{OPT}(G)
    $$
    where $F = 1 - 1/e \approx 0.62$.
\end{theorem}

\begin{proof}
    Consider the convex relaxation of the integral matching LP where $X_e$ is the indicator that is equal to 1 when $e$ is included in the matching.
    $$
    \begin{array}{llll}
        \text{maximize} & \sum_{e \in E} X_e \\
        \text{subject to} & \sum_{u \in N(v)} X_{uv} \leq 1 & \forall v \in V \\
        & \sum_{v \in N(u)} X_{uv} \leq 1 & \forall u \in U \\
        & X_e \geq 0 & \forall e \in E
    \end{array}
    $$
    The LP has $|U|+|V|$ variables and $|E|$ constraints. Take the dual of the LP. The dual is the vertex cover LP.
    $$
    \begin{array}{llll}
        \text{minimize} & \sum_{u \in U} \alpha_u^* + \sum_{v \in V} \beta_v^* \\
        \text{subject to} & \alpha_u^* + \beta_v^* \geq 1 & \forall uv \in E \\
        & \alpha_u^*,\beta_v^* \geq 0 & \forall u \in U,\, v \in V
    \end{array}
    $$
    $\alpha_u^*$ is the indicator variable that is 1 when $u$ is in the vertex cover. By weak duality of LP,
    $$
    \mathrm{LPOPT}(G) \leq \mathrm{DualLPOPT}(G)
    $$
    Further, for our relaxation,
    $$
    \mathrm{OPT}(G) \leq \mathrm{LPOPT}(G)
    $$
    Proof Idea: As the raning algorithm executes, we will construct a primal LP solution and a dual LP solution such that:
    \begin{enumerate}
        \item primal: is just the integral matching
        \item dual: fix $u \in U$, $v \in V$; if the ranking algorithm matches $v$ to $u$ as the online vertex $v$ arrives, then we set $\alpha_u = g(X_u)/F$ and $\beta_u = (1-g(X_u))/F$ where $F=1-1/e$ and $g(x) = -\exp(x-1)$; otherwise, set $\alpha_u = 0$ and $\beta_v = 0$ if $u$ and $v$ are unmatched. 
    \end{enumerate}
    When one more edge is admitted to $\mathcal{M}$, the objective function of the dual increased by $1/F > 1$.

    $(\alpha_u)_{u \in U}$ and $(\beta_v)_{v \in V}$ are random variables dependent on $(X_u)_{u \in U}$. Similarly, $\mathcal{M}$ is a random matching of $G$. The goal of the proof is to show that
    $$
    \mathbb{E}[|\mathcal{M}|] \geq (1-1/e) \cdot \mathrm{OPT}(G).
    $$
    Taking expectation of on both sides of \ref{eq:online-matching-objective}, we have
    $$
    \mathbb{E}[|\mathcal{M}|] = F \cdot \left[ \sum_{u \in U} \mathbb{E}[\alpha_u] + \sum_{v \in V} \mathbb{E}[\beta_v] \right]
    $$
    by linearity of expectation. Further, recall that $\mathrm{OPT}(G) \leq \mathrm{LPOPT}(G) \leq \mathrm{DualLPOPT}(G)$. Then for $\alpha_u^*$ and $\beta_v^*$ or any other feasible solution, 
    $$
    \mathrm{OPT}(G) \leq \mathrm{LPOPT}(G) \leq \mathrm{DualLPOPT}(G) \leq \sum_{u \in U} \alpha_u^* + \sum_{v \in V}\beta_v^*
    $$
    because the dual is a minimizing LP. Hence, it suffices to prove that for $\alpha_u^* = \mathbb{E}[\alpha_u]$ and $\beta_v^* = \mathbb{E}[\beta_v]$, $(\alpha_u^*)_{u \in U}$ and $(\beta_v^*)_{v \in V}$ are feasible solution for the dual LP. We now proceed with the proof.

    Fix an arbitrary edge $e_0 = (u_0,v_0) \in E$. We want to show that $\alpha_{u_0}^*$ and $\beta_{v_0}^*$ are feasible, which we can argue by showing $\alpha_{u_0}^* + \beta_{v_0}^* \geq 1$. Let $\tilde{G} = G \setminus \{u_0\}$. Fix $x_u$ for $u \neq u_0$. Consdier an execution of the ranking algorithm on $\tilde{G}$. Define $X^c$ to be the ``critical rank'' such that
    $$
    x^c = \begin{cases}
        X_u & \text{if $v_0$ matched to some $u \in U \setminus \{u_0\}$ } \\
        1 & \text{otherwise}
    \end{cases}
    $$

    \begin{lemma}[Dominance Lemma]
        If all ranks $(X_u)_{u \neq u_0}$ are fixed, then $u_0$ will be matched when the ranking algorithm executes on $G$, provided $X_{u_0} < X^c$.
    \end{lemma}

    \begin{lemma}[Monotonicity Lemma]
        Set $\beta_{v_0}^c = (1-g(x^c)) / F$. Think of this as the ``worst possible utility $v_0$ can gain''. If all r.v. $(X_u)_{u \neq u_0}$ are fixed, then when the ranking algorithm executes on $G$, we get that
        $$
        \beta_{v_0} \geq \beta_{v_0}^c.
        $$
    \end{lemma}

    Now, we prove the main claim. Consider $\beta_{v_0}$ first. By the Monotonicity Lemma, $\beta_{v_0} \geq \beta_{v_0}^c = (1-g(x^c)) / F$. Now consider $\alpha_{u_0}$ conditioned on $(X_u)_{u \neq u_0}$. Recall that $X_{u_0}$ is uniformly chosen from $[0,1]$ so $\mathcal{U}[0,1]$.
    $$
    \mathbb{E}\left[ \alpha_{u_0} \mid (X_u)_{u \neq u_0} \right] \geq \int_{0}^{x^c} g(s) ds \cdot \frac{1}{F}
    $$
    With this, we have the following bound
    $$
    \mathbb{E}\left[ \alpha_{u_0} \mid (X_u)_{u \neq u_0} \right] + \mathbb{E}\left[ \beta_{v_0} \mid (X_u)_{u \neq u_0} \right] \geq \int_{0}^{x^c} g(s) ds \cdot \frac{1}{F} + \frac{1-g(x^c)}{F}
    $$
    For any real $\theta \in [0,1]$,
    $$
    \int_0^\theta g(s) ds + (1-g(\theta)) \geq F = 1-\frac{1}{e}
    $$
    which gives a lower bound on $\mathbb{E}\left[ \alpha_{u_0} \mid (X_u)_{u \neq u_0} \right] + \mathbb{E}\left[ \beta_{v_0} \mid (X_u)_{u \neq u_0} \right]$ that is
    $$
    \mathbb{E}\left[ \alpha_{u_0} \mid (X_u)_{u \neq u_0} \right] + \mathbb{E}\left[ \beta_{v_0} \mid (X_u)_{u \neq u_0} \right] \geq \int_{0}^{x^c} g(s) ds \cdot \frac{1}{F} + \frac{1-g(x^c)}{F} \geq 1.
    $$
    Taking expectation on both sides again,
    $$
    \alpha_{u_0}^* + \beta_{v_0}^* = \mathbb{E}\big[\mathbb{E}\left[ \alpha_{u_0} \mid (X_u)_{u \neq u_0} \right] + \mathbb{E}\left[ \beta_{v_0} \mid (X_u)_{u \neq u_0} \right]\big] \geq 1.
    $$
\end{proof}
\end{document}
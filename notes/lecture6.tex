%
% This is the LaTeX template file for lecture notes for CS294-8,
% Computational Biology for Computer Scientists.  When preparing 
% LaTeX notes for this class, please use this template.
%
% To familiarize yourself with this template, the body contains
% some examples of its use.  Look them over.  Then you can
% run LaTeX on this file.  After you have LaTeXed this file then
% you can look over the result either by printing it out with
% dvips or using xdvi.
%
% This template is based on the template for Prof. Sinclair's CS 270.

\documentclass{article}
\usepackage{graphics}
\usepackage{multicol}
\usepackage{tikz}
\usepackage{clrscode3e}
\usepackage{enumitem}
\usepackage{amsmath}
\usepackage{amsthm}
\usepackage{amssymb}

\setlength{\oddsidemargin}{0 in}
\setlength{\evensidemargin}{0 in}
\setlength{\topmargin}{-0.6 in}
\setlength{\textwidth}{6.5 in}
\setlength{\textheight}{8.5 in}
\setlength{\headsep}{0.75 in}
\setlength{\parindent}{0 in}
\setlength{\parskip}{0.1 in}

%
% The following commands set up the lecnum (lecture number)
% counter and make various numbering schemes work relative
% to the lecture number.
%
\newcounter{lecnum}
\renewcommand{\thepage}{\thelecnum-\arabic{page}}
\renewcommand{\thesection}{\thelecnum.\arabic{section}}
\renewcommand{\theequation}{\thelecnum.\arabic{equation}}
\renewcommand{\thefigure}{\thelecnum.\arabic{figure}}
\renewcommand{\thetable}{\thelecnum.\arabic{table}}

%
% The following macro is used to generate the header.
%
\newcommand{\lecture}[4]{
   \pagestyle{myheadings}
   \thispagestyle{plain}
   \newpage
   \setcounter{lecnum}{#1}
   \setcounter{page}{1}
   \noindent
   \begin{center}
   \framebox{
      \vbox{\vspace{2mm}
    \hbox to 6.28in { {\bf CSC2420 - Algorithm Design, Analysis and Theory
                        \hfill Fall 2022} }
       \vspace{4mm}
       \hbox to 6.28in { {\Large \hfill Lecture #1: #2  \hfill} }
       \vspace{2mm}
       \hbox to 6.28in { {\it Lecturer: #3 \hfill Scribe: #4} }
      \vspace{2mm}}
   }
   \end{center}
   \markboth{Lecture #1: #2}{Lecture #1: #2}
   % {\bf Disclaimer}: {\it These notes have not been subjected to the
   % usual scrutiny reserved for formal publications.  They may be distributed
   % outside this class only with the permission of the Instructor.}
   \vspace*{4mm}
}

%
% Convention for citations is authors' initials followed by the year.
% For example, to cite a paper by Leighton and Maggs you would type
% \cite{LM89}, and to cite a paper by Strassen you would type \cite{S69}.
% (To avoid bibliography problems, for now we redefine the \cite command.)
% Also commands that create a suitable format for the reference list.
\renewcommand{\cite}[1]{[#1]}
\def\beginrefs{\begin{list}%
        {[\arabic{equation}]}{\usecounter{equation}
         \setlength{\leftmargin}{2.0truecm}\setlength{\labelsep}{0.4truecm}%
         \setlength{\labelwidth}{1.6truecm}}}
\def\endrefs{\end{list}}
\def\bibentry#1{\item[\hbox{[#1]}]}

%Use this command for a figure; it puts a figure in wherever you want it.
%usage: \fig{NUMBER}{SPACE-IN-INCHES}{CAPTION}
\newcommand{\fig}[3]{
			\vspace{#2}
			\begin{center}
			Figure \thelecnum.#1:~#3
			\end{center}
	}
% Use these for theorems, lemmas, proofs, etc.
\newtheorem{theorem}{Theorem}[lecnum]
\newtheorem{lemma}[theorem]{Lemma}
\newtheorem{proposition}[theorem]{Proposition}
\newtheorem{claim}[theorem]{Claim}
\newtheorem{corollary}[theorem]{Corollary}
\newtheorem{definition}[theorem]{Definition}
\newtheorem{problem}[theorem]{Problem}
\newtheorem{conjecture}[theorem]{Conjecture}
\renewenvironment{proof}{{\bf Proof:}}{\hfill\rule{2mm}{2mm}}

\newcommand{\Exp}{\mathbb{E}}
\newcommand{\Var}{\mathrm{Var}}


\begin{document}
%FILL IN THE RIGHT INFO.
%\lecture{**LECTURE-NUMBER**}{**DATE**}{**LECTURER**}{**SCRIBE**}
\lecture{6}{Online Bipartite Matching and Secretary Problem}{Calum MacRury}{Kevin Gao}

\section{Online Bipartite Matching in ROM}

In the random order model (ROM), the nodes $V$ of $G$ are presented to $\mathcal{A}$ uniformly at random, so the matching $\mathcal{A}(G)$ returned by $\mathcal{A}$ is random.

Performance of $\mathcal{A}$, namely $\Exp[w(\mathcal{A}(G))]$, is averaged over $\pi$. The competitive ratio of $\mathcal{A}$ in the random order model is then
$$
\inf_{G} \frac{\Exp[w(\mathcal{A}(G))]}{OPT(G)}
$$

Huang et al. 2018 introduced a generalization of Weighted-Ranking, defined using a function $g(x,y)$ where $g:\; [0,1]^2 \to [0,1]$.

The price of $u \in U$ is then $w_u \cdot g(X_u,Y_V)$ where $X_U \sim \mathcal{U}[0,1]$ is the rank of $u \in U$ and $Y_V \sim \mathcal{U}[0,1]$ is the arrival time of vertex $v \in V$.

For fixed $x \in [0,1]$, if $y_1 < y_2$, then $g(x,y_2) < g(x,y_1)$. Thus, the later a vertex $v$ arrives, the larger $Y_V$ is, the lower the prices for $v$.

\begin{theorem}
    Generalized-Ranking achieves a competitive ratio of $0.6534$ in the vertex-weighted ROM setting.
\end{theorem}

\section{Online Matching with Edge Weight}

We have so far only considered the case when $G = (U,V,E)$ is vertex weighted. All the arrival models and corresponding competitive ratios generalize to the edge weighted setting.

However, in the adversarial arrival model, no algorithm attains a constant competitive ratio. This is true even when $|U|=1$ and $|V|=2$. Consider the case when $w_{e_1} << w_{e_2}$.

\subsection{The Secretary Problem}

When $|U| = 1$, $E = \{u\} \times V$, observe that $OPT(G) = \max_{e \in E} w_e$. This is the secretary problem. Moreover, the algorithm can select at most one edge, and the edges arrive uniformly at random. The secretary algorithm is simple:

\begin{codebox}
    \Procname{$\proc{Secretary}$}
    \li pass on the first $n / e$ arriving edges
    \li accept the first edge whose weight is at least as large as the first $n / e$ edges
\end{codebox}

If the max edge is in the first $n / e$ edges, we don't pick anything. The probability of this happening is $1 / e$.

\begin{problem}
    \hfill
    
    \normalfont \textbf{Input}: Let there be $n$ applicants, say $1,\ldots,n$, where $w_i$ is the ``value/weight'' of applicant $i$.

    \textbf{Goal}: Output $I \in \{1,\ldots,n\}$ such that
    $$
    \Exp[w_I] \geq \alpha \cdot \max_{1 \leq i \leq n} w_i
    $$
\end{problem}

In the following pseudocode, $\pi$ is a random permutation of the applicants.

\begin{codebox}
    \Procname{$\proc{Threshold-Secretary}$}
    \li \For $t = 1$ \To $c\cdot n - 1$ \Do
        \li pass on applicant $\pi(t)$
    \End
    \li \For $t = c \cdot n$ \To $n$ \Do
        \li \If $w_{\pi(t)} \geq \max \{w_{\pi(1),\ldots,w_{\pi(cn-1)}}\}$ \Then
            \li \Return $w_{\pi(t)}$
\end{codebox}

\begin{theorem}
    If $j = \arg\max_{1 \leq i \leq n} w_i$,
    $$
    \Pr[I = i] \geq \alpha
    $$
    where $\alpha = 1/e$.
\end{theorem}
\begin{proof}
    Let $I$ be the output of the $cn$ threshold. We would like to find a lower bound on $\Pr[w_I = \max\{w_1,\ldots,w_n\}]$. This probability can be expressed as
    \begin{equation}
        \Pr[w_I = \max\{w_1,\ldots,w_n\}] = \sum_{t=1}^{n} \Pr[I = \pi(t) \land w_{\pi(t)} = \max\{w_1,\ldots,w_n\}]
    \end{equation}
    Observation: $\pi(t)$ is selected if and only if $\pi(t)$ is the best amongst the first $cn$ candidates and all candidates from $\pi(cn)$ to $\pi(t-1)$ are rejected. Equivalently, this is equal to the probability that $\pi(t)$ is the best applicant AND $\pi(cn),\ldots,\pi(t-1)$ are rejected.
    $$
    \begin{aligned}
        \Pr[I = \pi(t) \land \text{$\pi(t)$ is the best}] &= \Pr[\text{$\pi(cn),\ldots,\pi(t-1)$ rejected} \mid \text{$\pi(t)$ is the best}] \cdot \Pr[\text{$\pi(t)$ is the best}] \\
        &= \Pr[\text{$\pi(cn),\ldots,\pi(t-1)$ rejected} \mid \text{$\pi(t)$ is the best}] \cdot \frac{1}{n}
    \end{aligned}
    $$
    For every job $\pi(j)$ between $\pi(cn)$ and $\pi(t-1)$, it is rejected if and only if $w_{\pi(j)} < \max\{w_{\pi(1) \ldots w_{\pi(cn-1)}}\}$. This transforms the event of ``all jobs rejected'' into ``all jobs have value smaller than the max of the first $cn-1$ jobs'', which is something that we can calculate.
    $$
    \begin{aligned}
        \Pr\left[ \bigcap_{j=cn}^{t-1} \left( w_{\pi(j)} < \max\left\{w_{\pi(1),\ldots,w_{\pi(cn-1)}}\right\} \right) \right] &= \Pr\Bigg[ \parbox{14em}{ best applicant amongst the first $t-1$ arrivals in the first $cn-1$} \Bigg] \\
        &= \frac{cn-1}{t-1}
    \end{aligned}
    $$
    Substitute this result back to (5.1).
    \begin{equation}
        \Pr[w_I = \max\{w_1,\ldots,w_n\}] = \sum_{t=cn}^{n} \frac{cn-1}{t-1} \cdot \frac{1}{n}
    \end{equation}
    As $n \to \infty$, (5.2) can be upper bounded as follows.
    $$
    \begin{aligned}
        \sum_{t=cn}^{n} \frac{cn-1}{t-1} \cdot \frac{1}{n} &= (1+o(1)) c \cdot \sum_{t=cn}^{n} \frac{1}{t-1} \\
        &\leq (1 + o(1))c \int_{s=cn}^n \frac{1}{s} ds \\
        &= (1+o(1))c \cdot \left[ \ln(n) - \ln(cn) \right] \\
        &= c \ln c
    \end{aligned}
    $$
    This attains maximum for $c = \frac{1}{e}$.
\end{proof}

It turns out this ratio can be achieved in the general case where we have more than one offline vertices as well. The algorithm is as follows.

\begin{codebox}
    \Procname{$\proc{Secretary-Matching}$}
    \li $\mathcal{M} = \emptyset$
    \li $G_0 = (U,\emptyset,\emptyset)$
    \li \For $t = 1,\ldots,n$ \Do
        \li input $v_t$ and compute $G_t$ by updating $G_{t-1}$ to contain $v_t$
        \li \If $t < \lfloor n/e \rfloor$ \Then
            \li pass on $v_t$
        \li \Else
            \li compute an optimal matching $\mathcal{M}_t$ of $G_t$
            \li $e_t = $ the edge matched to $v_t$ via $\mathcal{M}_t$
            \li \If $e_t = (u_t, v_t)$ exists and $u_t$ is unmatched \Then
                \li $\mathcal{M} = \mathcal{M} \cup \{e_t\}$
            \End
        \End
    \End
    \li \Return $\mathcal{M}$  
\end{codebox}

\begin{theorem}[Kesselheim 2013, ESA]
    Secretary-Matching attains an asymptotic as $|V| \to \infty$ competitive ratio of $1/e$ for the case $|U| > 1$.
\end{theorem}

\end{document}
%
% This is the LaTeX template file for lecture notes for CS294-8,
% Computational Biology for Computer Scientists.  When preparing 
% LaTeX notes for this class, please use this template.
%
% To familiarize yourself with this template, the body contains
% some examples of its use.  Look them over.  Then you can
% run LaTeX on this file.  After you have LaTeXed this file then
% you can look over the result either by printing it out with
% dvips or using xdvi.
%
% This template is based on the template for Prof. Sinclair's CS 270.

\documentclass{article}
\usepackage{graphics}
\usepackage{multicol}
\usepackage{tikz}
\usepackage{clrscode3e}
\usepackage{enumitem}
\usepackage{amsmath}
\usepackage{amsthm}
\usepackage{amssymb}

\setlength{\oddsidemargin}{0 in}
\setlength{\evensidemargin}{0 in}
\setlength{\topmargin}{-0.6 in}
\setlength{\textwidth}{6.5 in}
\setlength{\textheight}{8.5 in}
\setlength{\headsep}{0.75 in}
\setlength{\parindent}{0 in}
\setlength{\parskip}{0.1 in}

%
% The following commands set up the lecnum (lecture number)
% counter and make various numbering schemes work relative
% to the lecture number.
%
\newcounter{lecnum}
\renewcommand{\thepage}{\thelecnum-\arabic{page}}
\renewcommand{\thesection}{\thelecnum.\arabic{section}}
\renewcommand{\theequation}{\thelecnum.\arabic{equation}}
\renewcommand{\thefigure}{\thelecnum.\arabic{figure}}
\renewcommand{\thetable}{\thelecnum.\arabic{table}}

%
% The following macro is used to generate the header.
%
\newcommand{\lecture}[4]{
   \pagestyle{myheadings}
   \thispagestyle{plain}
   \newpage
   \setcounter{lecnum}{#1}
   \setcounter{page}{1}
   \noindent
   \begin{center}
   \framebox{
      \vbox{\vspace{2mm}
    \hbox to 6.28in { {\bf CSC2420 - Algorithm Design, Analysis and Theory
                        \hfill Fall 2022} }
       \vspace{4mm}
       \hbox to 6.28in { {\Large \hfill Lecture #1: #2  \hfill} }
       \vspace{2mm}
       \hbox to 6.28in { {\it Lecturer: #3 \hfill Scribe: #4} }
      \vspace{2mm}}
   }
   \end{center}
   \markboth{Lecture #1: #2}{Lecture #1: #2}
   % {\bf Disclaimer}: {\it These notes have not been subjected to the
   % usual scrutiny reserved for formal publications.  They may be distributed
   % outside this class only with the permission of the Instructor.}
   \vspace*{4mm}
}

%
% Convention for citations is authors' initials followed by the year.
% For example, to cite a paper by Leighton and Maggs you would type
% \cite{LM89}, and to cite a paper by Strassen you would type \cite{S69}.
% (To avoid bibliography problems, for now we redefine the \cite command.)
% Also commands that create a suitable format for the reference list.
\renewcommand{\cite}[1]{[#1]}
\def\beginrefs{\begin{list}%
        {[\arabic{equation}]}{\usecounter{equation}
         \setlength{\leftmargin}{2.0truecm}\setlength{\labelsep}{0.4truecm}%
         \setlength{\labelwidth}{1.6truecm}}}
\def\endrefs{\end{list}}
\def\bibentry#1{\item[\hbox{[#1]}]}

%Use this command for a figure; it puts a figure in wherever you want it.
%usage: \fig{NUMBER}{SPACE-IN-INCHES}{CAPTION}
\newcommand{\fig}[3]{
			\vspace{#2}
			\begin{center}
			Figure \thelecnum.#1:~#3
			\end{center}
	}
% Use these for theorems, lemmas, proofs, etc.
\newtheorem{theorem}{Theorem}[lecnum]
\newtheorem{lemma}[theorem]{Lemma}
\newtheorem{proposition}[theorem]{Proposition}
\newtheorem{claim}[theorem]{Claim}
\newtheorem{corollary}[theorem]{Corollary}
\newtheorem{definition}[theorem]{Definition}
\newtheorem{problem}[theorem]{Problem}
\newtheorem{conjecture}[theorem]{Conjecture}
\renewenvironment{proof}{{\bf Proof:}}{\hfill\rule{2mm}{2mm}}

\newcommand{\Exp}{\mathbb{E}}
\newcommand{\Var}{\mathrm{Var}}


\begin{document}
%FILL IN THE RIGHT INFO.
%\lecture{**LECTURE-NUMBER**}{**DATE**}{**LECTURER**}{**SCRIBE**}
\lecture{2}{Approximation Algorithms Review}{Allan Borodin}{Kevin Gao}

\section{Online Bipartite Matching}

In a graph $G = (V,E)$, a matching is $M=(V,E')$ where $E' \subseteq E$ such that for every vertex $v$, the degree of $v$, $\deg_M(v)$ is at most 1. In an unweighted graph, the goal is to maximize the number of edges in the match. In an edge-weighted graph, the objective is to maximize the sum of weights of edges in the match. In the offline setting, the currently best result is Jack Edmonds's polynomial algorithm.

In a bipartite graph $G = (U \cup V,E)$, the vertices are partitioned into two sets $U,V$ with $U \cap V = \emptyset$ and $E \subseteq U \times V$.

In terms of online algorithms, the most common online model is one where one set of the vertices $V$ are known in advance, and the other isde of hte vertices $U$, arrive \textbf{online along with their adjacent edges}. The natural greedy approach would be to admit an edge whenever possible. This gives an $1/2$-approximation algorithm. This is the best deterministic algorithm.

There exists a randomized algorithm that achieves the competitive ratio $1-1/e$. The algorithm, known as the \textbf{Ranking Algorithm}, works by randomly assigning priorities to the offline vertices. Then when the online vertices arrive, we match them to the available offline vertex with the highest priority. 

Almost all the optimization problems we study can be formulated as \textbf{IPs}. These IPs can be relaxed to LPs. For example, the bipartite matching problem can be formulated as

$$
\begin{array}{lll}
    \text{maximize} & \displaystyle{\sum_{(u_i,v_j) \in E}} x_{ij} \\
    \text{subject to} & \displaystyle{\sum_{j:(u_i,v_j) \in E}} x_{i,j} \leq 1 & u_i \in V \\
    & \displaystyle{\sum_{i:(v_i,v_j) \in E}} x_{i,j} \leq 1 & v_j \in V \\
    & x_{i,j} \geq 0 & (u_i,v_j) \in E
\end{array}
$$

The Ranking Algorithm can also be viewed as a deterministic algorithm in the ROM (random order model) model. However, it is not known if the $1-1/e$ competitive ratio is optimal under the ROM model.

\subsection{AdWords}

In AdWords, edge weights represent bids $b_{ij}$ by an offline advertiser $v_j$ for a query impression $u_i$. Each offline advertiser $v_j$ has a budget $B_j$. The objective is to find a matching $M$ so as to maximize the following objective:
$$
\sum_{j} \min \left\{B_j, \sum_{i:e_{ij} \in M} b_{ij}\right\}
$$
i.e. Each advertiser does not generate mmore revenue than its budget.

\section{Makespan}

\subsection{Partial Enumeration Greedy}

Recall the makespan problem and the \textbf{LPT (longest processing time)} approximation algorithm. We can improve the $\frac{4}{3} - \frac{1}{3m}$ competitive ratio by solving a portion of the problem using brute-force search.

\begin{definition}
    Optimally schedule the largest $k$ jobs (for $0 \leq k \leq n$) and then greedily schedule the remaining jobs in any order.
\end{definition}

The algorithm has approximation ratio is no worse than $(1 + \frac{1-1/m}{1 + \lfloor k/m \rfloor})$. The running time is $O(m^k + n \log n)$. Setting $k = (1-\epsilon)/\epsilon$ gives a ratio of at most $(1+\epsilon)$. This is a PTAS with time $O(m^{m/\epsilon} + n\log n)$.

Online is not necessarily greedy. The best known approximation ratio is 1.9201 (Fleischer and Wahl) and there is 1.88 inapproximation bound (Rudin). Basic idea of some of the better algorithm is to leave some room for a possible large job; this forces the online algorithm to be \textbf{non-greedy} in some sense but still within the online model.

\subsection{Restricted Machine Model}

In the \textbf{restricted machine model}, every job $J_j$ is described by a pair $(p_j, S_j)$ where $S_j \subseteq \{1,\ldots,m\}$ is the set of machines on which $J_j$ can be scheduled. If each job has two allowable machine, this is equivalent to the graph orientation problem.

Azar et al. showed that $\log_2 (m)$ is the best competitive ratio for deterministic online algorithms with the upper bounds obtained by the ``natural greedy algorithm''. For randomized online, the ratio is $\ln(m)$.

\subsection{Unrelated Machine Model}

This is the most general case of the makespan machine models. Under this model, a job $J_i$ is represented by a vector $(p_{j,1},\ldots,p_{j,m})$ where $p_{j,i}$ is the time to process job $J_j$ on machine $i$. There is an online algorithm with approximation $O(\log m)$. This is currently the best-known approximation for greedy-like (priority-based) algorithms even for the restricted machines model.

\subsection{Precedence Constraints}

Graham also considered the makespan problem on identical machines for jobs satisfying a precedence constraint. Suppose that $\prec$ is a partial ordeer on jobs. If $J_i \prec J_k$, then $J_i$ must complete before $J_k$ can be started. Assuming jobs are ordered to respect the partial order, Graham showed that the ratio $2-1/m$ is achieved by the ``natural greedy algorithm'', call it $\mathcal{G}_\prec$.

Counterintuitively, many seemingly beneficial relaxations could actually lead to increase in the objective function that we want to minimize in makespan:
\begin{itemize}
    \item relaxing the precedence $\prec$
    \item decreasing the processing time of certain jobs
    \item adding more machines
\end{itemize}

\section{Knapsack}

\begin{definition}[Knapsack problem]
    \hfill \\
    \textbf{Input}: Knapsack size capacity $C$ and $n$ items $\mathcal{I} = \{I_1,\ldots,I_n\}$ where $I_j = (v_j,s_j)$. $v$ is the profit and $s$ is the size of the item. \\
    \textbf{Output}: A feasible subset $S \subseteq \{1,\ldots,n\}$ satisfying $\sum_{j \in S} s_j \leq C$ that maximizes $V(S) = \sum_{j\in S} v_j$.
\end{definition}

For some problems, approximation ratio is denoted with a value $\geq 1$. Many people (and literatures) use approximation ratios $\rho \leq 1$ for maximization problems. In this case, $\mathrm{ALG} \geq \rho \mathrm{OPT}$.

For the purpose of the knapsack problem, we generally restrict ourselves to the offline setting. The algorithm could still be ``myopic''. There have been studies done on the online version of the problem.

The partial enumeration greedy algorithm:

\begin{codebox}
    \Procname{$\proc{Partial-Enum-Knapsack}(\mathcal{I} = \{(v_1,s_1),\ldots,(v_n,s_n)\})$}
    \li sort $\mathcal{I}$ in non-increasing order by $v_i/s_i$
    \li \For every feasible subset $H \subseteq I$ with $|H| \leq k$ \Do
        \li $R = \mathcal{I} - H$
        \li $\mathrm{OPT}_H =$ optimal solution for $H$
        \li consider items in $R$ in order of $v_i/s_i$
        \li greedily add items to $\mathrm{OPT}_H$ not exceeding knapsack capacity $C$
    \End
    \li \Return $\mathrm{OPT}_H$ with max profit
\end{codebox}

\begin{theorem}[Sahni 1975]
    $V(\mathrm{OPT}) \leq (1+1/k)V(\proc{Partial-Enum-Knapsack})$. The algorithms runs in $kn^k$ time and setting $k=1/\epsilon$ gives an $(1+\epsilon)$ approximation algorithm.
\end{theorem}

\section{Priority Algorithm Model}

What is the intuitive nature of a greedy algorithm? With the exception of Huffman coding, all these algorithms consider one input item in each iteration and make an \textbf{irrevocable greedy decision} about that item.

\begin{codebox}
    \li $\mathcal{J} =$ set of all possible inputs
    \li $\preceq\,\, = $ a total ordering on $\mathcal{J}$ (typically induced by a function $f$)
    \li $\mathcal{I} \subset \mathcal{J} = $ actual input to the algorithm
    \li $S = \emptyset$ \RComment{items already examined by the algorithm}
    \li $i = 0$
    \li \While $\mathcal{I} - S \neq \emptyset$ \Do
        \li $i = i + 1$
        \li $\mathcal{I} = \mathcal{I} - S$
        \li $I_i = \min_{\preceq} \{ I \in \mathcal{I} \}$ \RComment{select min element based on the ordering $\preceq$}
        \li make an irrevocable decision $D_i$ concerning $I_i$ 
        \li $S = S \cup \{I_i\}$
    \End      
\end{codebox}

Here, we make no assumption on the complexity or even the computability of the ordering $\preceq$ or function $f$.

This template can be modified to allow the algorithm to determine the total ordering dynamically based on the elements it has observed so far.

\begin{codebox}
    \li $\mathcal{J} =$ set of all possible inputs
    \li $\mathcal{I} \subset \mathcal{J} = $ actual input to the algorithm
    \li $S = \emptyset$ \RComment{items already examined by the algorithm}
    \li $i = 0$
    \li \While $\mathcal{I} - S \neq \emptyset$ \Do
        \li $i = i + 1$
        \li $\preceq_i\,\, = $ a total ordering on $\mathcal{J}$ (typically induced by a function $f_i$)
        \li $\mathcal{I} = \mathcal{I} - S$
        \li $I_i = \min_{\preceq_i} \{ I \in \mathcal{I} \}$ \RComment{select min element based on the ordering $\preceq_i$}
        \li make an irrevocable decision $D_i$ concerning $I_i$ 
        \li $S = S \cup \{I_i\}$
        \li $\mathcal{J} = \mathcal{J} - \{I \in \mathcal{I} \mid I \preceq_i I_i \}$
    \End      
\end{codebox}

\subsection{Inapproximation w.r.t. Priority Model}

\end{document}
%
% This is the LaTeX template file for lecture notes for CS294-8,
% Computational Biology for Computer Scientists.  When preparing 
% LaTeX notes for this class, please use this template.
%
% To familiarize yourself with this template, the body contains
% some examples of its use.  Look them over.  Then you can
% run LaTeX on this file.  After you have LaTeXed this file then
% you can look over the result either by printing it out with
% dvips or using xdvi.
%
% This template is based on the template for Prof. Sinclair's CS 270.

\documentclass{article}
\usepackage{graphics}
\usepackage{multicol}
\usepackage{tikz}
\usepackage{clrscode3e}
\usepackage{enumitem}
\usepackage{amsmath}
\usepackage{amsthm}
\usepackage{amssymb}

\setlength{\oddsidemargin}{0 in}
\setlength{\evensidemargin}{0 in}
\setlength{\topmargin}{-0.6 in}
\setlength{\textwidth}{6.5 in}
\setlength{\textheight}{8.5 in}
\setlength{\headsep}{0.75 in}
\setlength{\parindent}{0 in}
\setlength{\parskip}{0.1 in}

%
% The following commands set up the lecnum (lecture number)
% counter and make various numbering schemes work relative
% to the lecture number.
%
\newcounter{lecnum}
\renewcommand{\thepage}{\thelecnum-\arabic{page}}
\renewcommand{\thesection}{\thelecnum.\arabic{section}}
\renewcommand{\theequation}{\thelecnum.\arabic{equation}}
\renewcommand{\thefigure}{\thelecnum.\arabic{figure}}
\renewcommand{\thetable}{\thelecnum.\arabic{table}}

%
% The following macro is used to generate the header.
%
\newcommand{\lecture}[4]{
   \pagestyle{myheadings}
   \thispagestyle{plain}
   \newpage
   \setcounter{lecnum}{#1}
   \setcounter{page}{1}
   \noindent
   \begin{center}
   \framebox{
      \vbox{\vspace{2mm}
    \hbox to 6.28in { {\bf CSC2420 - Algorithm Design, Analysis and Theory
                        \hfill Fall 2022} }
       \vspace{4mm}
       \hbox to 6.28in { {\Large \hfill Lecture #1: #2  \hfill} }
       \vspace{2mm}
       \hbox to 6.28in { {\it Lecturer: #3 \hfill Scribe: #4} }
      \vspace{2mm}}
   }
   \end{center}
   \markboth{Lecture #1: #2}{Lecture #1: #2}
   % {\bf Disclaimer}: {\it These notes have not been subjected to the
   % usual scrutiny reserved for formal publications.  They may be distributed
   % outside this class only with the permission of the Instructor.}
   \vspace*{4mm}
}

%
% Convention for citations is authors' initials followed by the year.
% For example, to cite a paper by Leighton and Maggs you would type
% \cite{LM89}, and to cite a paper by Strassen you would type \cite{S69}.
% (To avoid bibliography problems, for now we redefine the \cite command.)
% Also commands that create a suitable format for the reference list.
\renewcommand{\cite}[1]{[#1]}
\def\beginrefs{\begin{list}%
        {[\arabic{equation}]}{\usecounter{equation}
         \setlength{\leftmargin}{2.0truecm}\setlength{\labelsep}{0.4truecm}%
         \setlength{\labelwidth}{1.6truecm}}}
\def\endrefs{\end{list}}
\def\bibentry#1{\item[\hbox{[#1]}]}

%Use this command for a figure; it puts a figure in wherever you want it.
%usage: \fig{NUMBER}{SPACE-IN-INCHES}{CAPTION}
\newcommand{\fig}[3]{
			\vspace{#2}
			\begin{center}
			Figure \thelecnum.#1:~#3
			\end{center}
	}
% Use these for theorems, lemmas, proofs, etc.
\newtheorem{theorem}{Theorem}[lecnum]
\newtheorem{lemma}[theorem]{Lemma}
\newtheorem{proposition}[theorem]{Proposition}
\newtheorem{claim}[theorem]{Claim}
\newtheorem{corollary}[theorem]{Corollary}
\newtheorem{definition}[theorem]{Definition}
\newtheorem{problem}[theorem]{Problem}
\newtheorem{conjecture}[theorem]{Conjecture}
\renewenvironment{proof}{{\bf Proof:}}{\hfill\rule{2mm}{2mm}}

\newcommand{\Exp}{\mathbb{E}}
\newcommand{\Var}{\mathrm{Var}}


\begin{document}
%FILL IN THE RIGHT INFO.
%\lecture{**LECTURE-NUMBER**}{**DATE**}{**LECTURER**}{**SCRIBE**}
\lecture{7}{Local Search and Randomization}{Allan Borodin}{Kevin Gao}

\section{Facility Location}

\begin{problem}[(Uncapacitated) Metric Facility Location Problem]
    \hfill 

    \normalfont Input: $(F,C,d,f)$ where $F$ is a set of facilities, $C$ is a set of clients/cities, $d$ is a metric distance functoin over $F \cup C$, and $f$ is an opening cost function for facilities.

    \normalfont Output: A subset of facilities $F'$ minimizing $\sum_{i \in F'} f_i + \sum_{j \in C} d(j,F')$ where $f_i$ is the opening cost of facility $i \in F$ and $d(j,F') = \min_{i \in F'}d(j,i)$. 
\end{problem}

In the capacitated version, facilities have capacities and cities can have demands (rather than unit demand). The constraint is that a facility cannot have more assigned demand than its capacity so that it is not possible to always assign a city to its closest facility.

\begin{problem}[$k$-Median Problem]
    \hfill

    \normalfont \textbf{Input}: $(F,C,d,k)$ where $F,C,d$ are as in UFL and $k$ is the number of facilities that can be opened.

    \normalfont \textbf{Output}: A subset of facilities $F'$ with $|F'| = k$ that minimizes $\sum_{j \in C} d(j,F')$.
\end{problem}

UFL is hard to approximate to within a factor better than 1.463 assuming $\mathrm{P} \nsubseteq \mathrm{DTIME}(n^{\log\log n})$ (Guha, Khuller 1999). the $k$-Median problem is hard to approximate to within a factor of $1 + 1/e \approx 1.736$ (Jain, Mahdian, Saberi). The current best polynomial time approximation for UFL is $1.488$ (Li, 2011).

\section{k-independence Systems}

There are many ways to extend matroids. In particular, the exchange property immediately implies that in a matroid $M$, every maximal independent set (called a base) has the same cardinality, the rank of $M$. Matroids are those independence systems where all bases have the same cardinality. Let $k$ be a positive integer. A (Jenkyns) k-independence system satisfies the weaker property that for any set $S$ and two bases $B$ and $B_0$ of $S$, $|B|/|B_0| \leq k$. When $k=1$, the system is a matroid. An example of a $k$-independent system is the intersection of $k$ matroids.

\section{Submodular Functions}

Let $U$ be a universe. We consider the set of functions $f(S) \geq 0$ for all $S \subseteq U$. We will also assume that the functions are normalized in that $F(\emptyset) = 0$.

A \textbf{sublinear} set function satisfies the property that
$$
f(S \cup T) \leq f(S) + f(T)
$$
for all subsets $S,T \subseteq U$.

When $f(S \cup T) + f(S \cap T) = f(S) + f(T)$, the function is a \textbf{linear} (or \textbf{modular}) function.

A \textbf{submodular} set function $f:\; U \to \mathbb{R}$ satisfies the property that
$$
f(S \cup T) + f(S \cap T) \leq f(S) + f(T)
$$
It follows that modular set functions are submodular and submodular functions sublinear. Submodular functions can be monotone or non-monotone. A monotone submodular function also satisfies the property that
$$
f(S) \leq f(T)
$$
whenever $S \subseteq T$.

Submodular function can be equivalently characterized as those satisfying the \textbf{decreasing marginal gain} property, where for $S \subset T$ 
$$
f(T \cup \{x\}) - f(T) \leq f(S \cup \{x\}) - f(S)
$$
That is, adding additional elements has non-increasing marginal gain for larger sets. In other words, we gain more by adding elements to a smaller set than to a larger set. This is also called \textbf{diminishing returns}: the more you have, the less you want.

Some observations:
\begin{itemize}
    \item Modular functions are monotone.
    \item The rank of a matroid is a monotone submodular function.
    \item The two most common examples of non-monotone submodular functions are max-cut and max-di-cut.
\end{itemize}

The monotone problem is only interesting when the submodular maximization is subject to some constraint. One of the simplest constraints is a cardinality constraint: to maximize $f(S)$ subject to $|S| \leq k$ for some $k$; since $f$ is monotone, this is equivalent to requiring $f(S) = k$. The following greedy algorithm for the constrained monotone function maximization problem with the approximation ratio $1 - (1-1/k)^k$ approximation for a $k$ cardinality constraint.

\begin{codebox}
    \li $S = \emptyset$
    \li \While $|S| \leq k$ \Do
        \li $u = \arg\max_u f(S \cup \{u\}) - f(S)$
        \li $S = S \cup \{u\}$
    \End
\end{codebox}

\begin{theorem}
    The above algorithm achieves an approximation ratio of $1 - (1 - 1/k)^k$ for monotone submodular function maximization subject to cardinality constraint.
\end{theorem}

\begin{proof}
    
\end{proof}

\section{Randomized Algorithms}

There are some problem settings (e.g. simulation, cryptography, interactive proofs, sublinear algorithms) where randomization is necessary. We can also use randomization to improve approximation ratios.

In complexity theory, a fundamental question is how much can randomization lower the time complexity of a problem. For decision problems, there are three polynomial time randomized classes ZPP (zero-sided), RP (one-sided), and BPP (two-sided) error. The big question in \textbf{fine-grained complexity theory} is $\mathrm{BPP} \overset{?}{=} \mathrm{P}$.

One important aspect of randomized algorithms in an offline setting is that the probability of success can be amplified by repeated independent trials of the algorithm.

\end{document}